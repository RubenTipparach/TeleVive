\documentclass[10pt,a4paper]{article}
\usepackage[latin1]{inputenc}
\usepackage{amsmath}
\usepackage{amsfonts}
\usepackage{amssymb}
\usepackage{makeidx}
\usepackage{graphicx}
\usepackage{url}

\usepackage{geometry}
\geometry{
	a4paper,
	total={170mm,257mm},
	left=20mm,
	top=20mm,
}

%opening
\title{\textbf{\Huge Enhancing the Virtual Experience Using Photogrammetry Techiques\\
		\Large with the HTC Vive }}
\author{Santipab Tipparach\\
	\textbf{North Dakota State University}\\\textbf{Department of Computer Science}}

\begin{document}
	\maketitle

	\begin{abstract}
	The age of virtual reality as mainstream media has finally dawned. With it, consumers everywhere are looking for innovations in immersive worlds, next-gen video games, and advanced human-computer interfaces. This proposal will study the viability of a data processing architecture using inexpensive Raspberry Pi's to construct a photogrammetric array to scan objects, places, and people into a virtual world. The experiment will consist of blending together still models and fully animated 3D models all displayed within a virtual environment as a holographic representation.

	\end{abstract}

	\section{Introduction}
	Virtual reality is the new frontier of human-computer interaction. Since the invention of the television, communication has expanded into the visual a motion picture field. VR is the next generation of display, it allows the user to become fully immersed in the virtual world. Today, virtual reality exists mostly in the video games, and marketing industries. There are some the seek to expand the field and build an industry servicing engineers and scientific data. One field of VR is yet to be conquered by any single entity, and that is user generated content area.
	\\\\
	This study proposes a cost effective and highly immersive architecture for a system that will enable users to capture, record, and replay spatail geometric and textual data in realtime. This means a virtual representation of a real world place, person, or thing; nouns in VR. The world of communication will be transformed into the illusion that is a byproduct of affordable destinations that can capture a human scale image and stream that data to another location. This is the idea behind virtual teleportation.
	\\\\
	By utilizing an array Rapberry Pi 3, PiCams, and the distributed nature of the Pi, this study aims at creating an infrastructure of inexpensive micro computers that can work together to quickly produce images that can be viewed almost instantly in a VR environment such as the HTC Vive. The main reason for the choice of the Vive is that it allows for room scale movement and tracking. This means that the Pi image array can also be room sized allowing for models of larger scales to be uploaded.

	\section{Background}
    The evolution of computer graphics have come a long way in the past 3 decades or so. With the advent of the GPU, and graphics dedicated memory, computers are able to render realtime images with high parallelism and increasing complexity. Photogrammetry is the method of scanning 3D objects from the physical world into the virtual\cite{Pisa}. This method has been used for various methods of construction virtual scenes, such as 3D printing a CAD model and detailing using physical artistic methods\cite{Swearingen}. Photogrammetry is also a tool used for preserving historical objects and sites \cite{Pisa}. It can also be used as a tool for education in the classroom \cite{Campbell}, using depth scanning systems like the Microsoft Kinect.
    \\\\
    A few years passed as advancements in computer graphics exploded at the beginning of the 21st century, advancements in display and micro sensors were made in the mobile computing market, the age of smart phones. These ushered in high precision devices that kept track of their GPS coordinates, accelerometers, and gyroscopic orientation. By the 2010s, virtual reality technology became viable, and relatively low cost for the general public. VR entered both the entertainment and enterprise industry almost simultaneously as 2016 brought in the year of consumer VR. With it, applications for the engineering and construction industry became an invaluable market \cite{Hilfert}. One that allowed complex data to be visualized in realtime. One of the largest challenges of VR is the psychological aspects of better understanding how this immersive environment affects the human brain. These studies include the comprehension of the vast nest of fully immersive virtual environments \cite{Rauhoeft}, the quality of 3D printed and scanned objects in comparison to their original 3D object \cite{Thorn}, and the comparison of virtual environments with their real world models\cite{Eckstein}. In addition, with the development of the HTC Vive, more precise measurements and data collection can be made using highly detailed and accurate room scale sensors\cite{Soffel}.
    \\\\
    Game development has been the primary driving force behind these innovative technologies. With the market rapidly growing every year, the quest for the ultimate simulation has been a long sought after goal of game developers and GPU-engineers alike. Balancing interactivity and influence have been key to finding the happy medium to telling an immersive story\cite{Craven} since the dawn of 3D games.
    \\\\
    As VR approaches the mainstream consumer, both applications and video games will be experienced in an exciting new medium. With it, comes the development of new tools, frameworks, and popularizing old techniques. Combining VR and photogrammetry will take the immersive to a whole new level of documenting the human experience. Telling stories along the way, and helping the entertainment and enterprise industries realize the true potential of what VR has to offer.

	\section*{REFERENCES}
	\begin{thebibliography}{99}

		% I'll have to look this up...
		\bibitem{Shinozuka}{Yukiko Shinozuka and Hideo Saito, ''Sharing 3D Object with Multiple Clients via Networks Using Vision-Based 3D Object Tracking'',2016.}

		%representations of worlds in unity vR
		\bibitem{Eckstein}{Benjamin Eckstein and Birgit Lugrin, ''Augmented Reasoning in the Mirror World'',2016.}

		%photogrammetry using kinect
		\bibitem{Campbell}{Dr. Abraham G. Campbell et. al., ''Future Mixed Reality Educational Spaces'',2016.}

        %A method for modeling environments using photogrammetry
        \bibitem{Swearingen}{Scott Swearingen And Kyoung Lee Swearingen, ''Creating Virtual Environments with 3D Printing and Photogrammetry'',2016.}

        %A method for scanning elements
        \bibitem{Pisa}{Cecilia Pisa et. al., ''Spherical Photogrammetry for Cultural Heritage San Galgano Abbey and the Roman Theater, Sabratha'',2011.}

        %Reason why Vive is cool
        \bibitem{Soffel}{Fabian Soffel et. al., ''Postural Stability Analysis in Virtual Reality Using the HTC Vive'',2016.}

        %How VR can solve practical problems
        \bibitem{Hilfert}{Thomas Hilfert and Markus K�onig, ''Low-cost virtual reality environment for
engineering and construction'',2016.}

        %VR story telling, interactivity vs influence
        \bibitem{Craven}{Mike Craven et. al., ''Exploiting Interactivity, Influence, Space and Time
to Explore Non-Linear Drama in Virtual Worlds'',2001.}

        %How VR can help understand space
        \bibitem{Rauhoeft}{Greg Rauhoeft et. al., ''Evoking and Assessing Vastness in Virtual Environments'',2015.}

        %Assessing 3D scan quality
        \bibitem{Thorn}{Jacob Thorn et. al., ''Assessing 3D Scan Quality Through Paired-comparisons Psychophysics'',2016.}

    	\end{thebibliography}
\end{document}
