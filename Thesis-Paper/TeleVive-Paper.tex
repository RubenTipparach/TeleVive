\documentclass[10pt,a4paper]{article}
\usepackage[latin1]{inputenc}
\usepackage{amsmath}
\usepackage{amsfonts}
\usepackage{amssymb}
\usepackage{makeidx}
\usepackage{graphicx}
\usepackage{url}

\usepackage{geometry}
\geometry{
	a4paper,
	total={170mm,257mm},
	left=20mm,
	top=20mm,
}

%opening
\title{\textbf{\Huge Enhancing the Virtual Experience Using Photogrammetry Techiques\\
		\Large with the HTC Vive }}
\author{Santipab Tipparach\\
	\textbf{North Dakota State University}\\\textbf{Department of Computer Science}}

\begin{document}
	\maketitle
	
	\begin{abstract}
	The age of virtual reality as mainstream media has finally dawned. With it, consumers everywhere are looking for innovations in immersive worlds, next-gen video games, and advanced human-computer interfaces. This proposal will study the viability of a data processing architecture using inexpensive Raspberry Pi's to construct a photogrammetric array to scan objects, places, and people into a virtual world. The experiment will consist of blending together still models and fully animated 3D models all displayed within a virtual environment as a holographic representation.
		
	\end{abstract}
	
	\section{Introduction}

	
	\section{Background}
	

\cite{Shinozuka}

	\section*{REFERENCES}
	\begin{thebibliography}{99}
		
		\bibitem{Shinozuka}{Yukiko Shinozuka and Hideo Saito, ''Sharing 3D Object with Multiple Clients via Networks
			Using Vision-Based 3D Object Tracking'',2016.}
		
	\end{thebibliography}
\end{document}
