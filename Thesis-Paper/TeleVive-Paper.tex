\documentclass[10pt,a4paper]{article}
\usepackage[latin1]{inputenc}
\usepackage{amsmath}
\usepackage{amsfonts}
\usepackage{amssymb}
\usepackage{makeidx}
\usepackage{graphicx}
\usepackage{url}

\usepackage{geometry}
\geometry{
	a4paper,
	total={170mm,257mm},
	left=20mm,
	top=20mm,
}

%opening
\title{\textbf{\Huge Enhancing the Virtual Experience Using Photogrammetry Techiques\\
		\Large with the HTC Vive }}
\author{Santipab Tipparach\\
	\textbf{North Dakota State University}\\\textbf{Department of Computer Science}}

\begin{document}
	\maketitle

	\begin{abstract}
	The age of virtual reality as mainstream media has finally dawned. With it, consumers everywhere are looking for innovations in immersive worlds, next-gen video games, and advanced human-computer interfaces. This project will study the viability of a data processing architecture using inexpensive Raspberry Pi's to construct a photogrammetric array to scan objects, places, and people into a virtual world. The experiment will consist of blending together still models and fully animated 3D models all displayed within a virtual environment as a holographic representation.

	\end{abstract}

	\section*{Introduction}
	Virtual reality is the new frontier of human-computer interaction. Since the invention of the television, communication has expanded into the visual motion picture field. VR is the next generation of display, it allows the user to become fully immersed in the virtual world. Today, virtual reality exists mostly in video games, and marketing industries. There are some that seek to expand the field and build an industry servicing engineers and scientists. One field of VR is yet to be conquered by any single entity, and that is the social media of user generated content such as Youtube or Instagram.
	\\\\
	This study proposes a cost effective and highly immersive architecture for a system that will enable users to capture, record, and replay spatial geometric and textured data in real time. This means a virtual representation of a real world place, person, or thing. The world of communication will be transformed into a database of affordable pre-captured destinations that can display a human scale 3D hologram and be able to stream that data across wide distances. This is the idea behind virtual teleportation.
	\\\\
	By utilizing an array of Rapberry Pi 3, PiCams, and the distributed computing nature of the Pi, this study aims at creating an infrastructure of inexpensive micro computers that can work together to quickly produce images that can be viewed almost instantly in a VR environment such as the HTC Vive. The main reason for the choice of the Vive is that it allows for room scale movement and tracking. This means that the Pi image array can also be room sized allowing for models of larger scales to be uploaded.

	\section*{Background}
    The evolution of computer graphics have come a long way in the past 3 decades or so. With the advent of the GPU, and graphics dedicated memory, computers are able to render real time images with high parallelism and increasing complexity. Photogrammetry is the method of scanning 3D objects from the physical world into the virtual\cite{Pisa}. This method has been used for various methods for the construction of virtual scenes, such as 3D printing a CAD model and detailing using physical artistic methods\cite{Swearingen}. Photogrammetry is also a tool used for preserving historical objects and sites \cite{Pisa}. It can also be used as a tool for education in the classroom \cite{Campbell}, using depth scanning systems like the Microsoft Kinect.
    \\\\
    A few years passed as advancements in computer graphics exploded at the beginning of the 21st century, advancements in display and micro sensors were made in the mobile computing market, the age of smart phones. These ushered in high precision devices that kept track of their GPS coordinates, accelerometers, and gyroscopic orientation. By the 2010s, virtual reality technology became viable, and relatively low cost for the general public. VR entered both the entertainment and enterprise industry almost simultaneously as 2016 brought in the year of consumer VR. With it, applications for the engineering and construction industry became an invaluable market \cite{Hilfert}. One that allowed complex data to be visualized in real time. One of the largest challenges of VR is the psychological aspects of better understanding how this immersive environment affects the human brain. These studies include the comprehension of the vastness of fully immersive virtual environments \cite{Rauhoeft}, the quality of 3D printed and scanned objects in comparison to their original 3D object \cite{Thorn}, and the comparison of virtual environments with their real world models\cite{Eckstein}. In addition, with the development of the HTC Vive, more precise measurements and data collection can be made using highly detailed and accurate room scale sensors\cite{Soffel}.
    \\\\
    Game development has been the primary driving force behind these innovative technologies. With the market rapidly growing every year, the quest for the ultimate simulation has been a long sought after goal of game developers and GPU-engineers alike. Balancing interactivity and influence have been key to finding the happy medium to telling an immersive story\cite{Craven} since the dawn of 3D games.
    \\\\
    As VR approaches the mainstream consumer, both applications and video games will be experienced in an exciting new medium. With it, comes the development of new tools, frameworks, and popularizing old techniques. Combining VR and photogrammetry will take the immersion to a whole new level of documenting the human experience. Telling stories along the way, and helping the entertainment and enterprise industries realize the true potential of what VR has to offer.

	\subsection*{History of VR}
	In 2016, over 12 headsets were released. This shows a trend of exponential growth of a relatively small market, that seems to be expanding at an alarming rate. Many investors in the industry are pouring large amounts of money into VR research. Virtual reality at its core concept of using stereographic images to display an interactive environment can be dated back to as far as the 1980s when NASA first developed VR to train astronauts for space missions, and to help operate mechanical robots in space.
	\\\\
	Towards the 90s, VR came to consumers with the Virtual Boy by Nintendo. This was meant as a platform to create new games on that would fully immerse the players. Although it didn't succeed in marketing, and selling 770,000 units with only 22 games published, it was regarded as a significant and bold step in VR development. The main issue with the Virtual Boy was that it only displayed in red LED with a resolution of 384x224 pixels. This was a very low resolution console that caused the public to lose interest in VR for quite some time.
	\\\\
	The next big push to VR would not arrive until 2012 with the announcement of the Oculus Rift company and its development kit, a true VR headset that utilized a cell phone screen inside of a box, connected to the PC. This also ushered in an age of mobile VR with the Google Cardboard. As these two products began hitting the market and small startups, investment in VR became a reality. Silicon Valley lit up with all kinds of VR technology development. By 2016, it became the year of VR.
	\\\\
	The Oculus Rift and the HTC Vive released within just a week with each other at the end of Q1 2016. These both saw moderate sales, both in roughly over half a million units. The Samsung VR selling the most by 2 million units, and the PS4 VR at nearly a million. Market experts have predicted that 2017 will be the year that VR content becomes ubiquitous. With a wide range of technologies available, the tech industry and research institutions alike should embrace this new medium as a way to visualize and interact with data in the future.
	\\\\
	The HTC Vive is the main focus of this thesis. Development began back in 2012 when Valve and HTC began working separately on different VR projects\cite{Volpe} \cite{Souppouris}. in wake of the announcement of the Oculus Rift's development and its affect on the entire tech industry when the famed Kickstarter campaign launched an entirely new industry in VR technology. By 2014, Valve's efforts in VR with Oulus came to a grinding halt as Oculus was aquired by Facebook. This led to Valve partnering with HTC to develop the ultimate room scale VR.
	\\\\ 	%april tags
	In early development, Valve used April tags, which are a simplified version of the QR tags to allow a headset with mounted LEDs to track room scale movement and display graphics in VR. This quickly became infeasible since the tracking wasn't precise enough. When Valve partnered with HTC to develop the Vive, the goal from the beginning was to develop a headset capable of room scale tracking and allowing for a high level of fidelity.
	\\\\
	Over the course of its 2 year development cycle, the two companies rapidly developed and prototyped many different kinds of headsets ranging from dot tracking to laser tracking. Laser tracking achieved the best result since it involves the headset picking up laser emitted dots from a sweeping diode that allowed it to calculate its distance from the base station. From this, the headset could determine its position and orientation in 3D space with a high degree of accuracy. Thus room scale tracking was achieved. Over its course of development Valve and HTC involved a number of companies that tried the product and provided a stream of feedback and networked together a small community of developer that drove the development in the direction HTC and Valve wanted.
	\\\\
	The product was then announced at the Mobile World Congress (MWC) and the Game Developer Conference(GDC) in 2015. The press announcement was overwhelmingly positive, with the Vive winning many awards. Its release in 2016 made it the CES product of the year. The Vive's room scale abilities make it the perfect tool for studying how room-scale VR coupled with room scale photogrammetry will make for a genuine immersive experience.
	
	
	\subsection*{History of Photogrammetry}
	Photogrammetry covers a wide range of disciplines in collecting data and converting it to a point cloud system that then allows the reconstruction of spatial data along with its textures. The method that will be discussed in this research mostly deals with the interactive media aspect of photogrammetry. Meaning the scanning of 3D objects at the scale that will allow users to interact with it.
	\\\\
	Although there is no clearly documented timescale for the capture of real life objects in video games, there is a clear trend that has been taking place in the last 20 or so years since the advent of GPU accelerated 3D rendering. This goes back to the late 90s and early 2000s when video cards had enough memory to process large volumes of texture data. This meant that polygons could be mapped to higher resolution images, and allow for more accurate looking objects.
	\\\\
	By the early 2000s games and 3D rendering technology had come a long, polygons could now be increased to a much higher amount. Although some games still had the character's faces and proportions incorrect. Laser scanning and photographic images helped to correct these errors by providing much more accurate 3D scanning data.
	\\\\
	Today 3D photogrammetry process are used to capture, in high detail, the likeness of celebrities and actors that would then be used in game as well as preserve aging ancient artifacts \cite{Samann}. For example, many of EA game's sports titles use real life soccer players to be scanned into the game to portray an extremely accurate representation. Another more recent game title, Hideo Kojima's Death Stranding shows actors being scanned by a custom built rig and some of the best experts in the field of human likeness capture photographers in the world. This process invloves dozens of cameras mounted in elaborately built lighting studios to maximize the amount of detail and accuracy captured.
	\\\\
	However, it is not only humans that have received the grand treatment. For some time, games have also been trying to capture entire environments. One example is Crysis in 2007, sent out a team of photographers to capture a pictures in remote islands in the south pacific to capture exotic plants and animals to model the in game tropical island environment. This was a costly process and involved hundreds of thousands of photographs to be manually processed by artists and developed into 3D assets for in game use.
	\\\\
	Another recent example of the video game photogrammetry is in Star Wars: Battlefront (2015) where artists were sent to remote environments like the volcanic and snowy terrariums of Iceland to capture some rare sceneries and textures to be recreated as assets for the game. This made Battlefront one of the best looking games of 2015, with nearly photo perfect realism coupled by a powerful rendering engine.
	\\\\
	With the photogrammetric techniques combined with low-cost Raspberry Pi tech, this research aims to lower the cost of capturing real life objects and environments and convert them into immersive experiences that can be viewed inside of VR \cite{Pi3DScan}.
	
	\subsection*{Raspberry Pi Scanner Arrays}
	To save costs on building expensive camera rigs or setups, the Raspbery Pi offers a compact camera and computing system that is low cost and efficient for processing images in parallel to create an instant photogrammetry array. Large rigs can costs anywhere from \$20,000 to \$200,000 \cite{Straub2}, therefore alternatives for research must be taken to cut down on costs and still be able to achieve relatively accurate results.
	\\\\
	Pi Scanner arrays can be assembled with relatively cheap materials such as lighting stands or metal poles where the Pi cameras can be mounted. After that, the arrangement must be so that the user can enter the area and be scanned from the photography. Once the subject is scanned the rendering or mesh building software can be used. This is often either Agisoft's 3D scan application or Autodesk Recap 360.
	
	\section*{REFERENCES}
	\begin{thebibliography}{99}

		% I'll have to look this up...
		\bibitem{Shinozuka}{Yukiko Shinozuka and Hideo Saito, ''Sharing 3D Object with Multiple Clients via Networks Using Vision-Based 3D Object Tracking'',2016.}

		%representations of worlds in unity vR
		\bibitem{Eckstein}{Benjamin Eckstein and Birgit Lugrin, ''Augmented Reasoning in the Mirror World'',2016.}

		%photogrammetry using kinect
		\bibitem{Campbell}{Dr. Abraham G. Campbell et. al., ''Future Mixed Reality Educational Spaces'',2016.}

        %A method for modeling environments using photogrammetry
        \bibitem{Swearingen}{Scott Swearingen And Kyoung Lee Swearingen, ''Creating Virtual Environments with 3D Printing and Photogrammetry'',2016.}

        %A method for scanning elements
        \bibitem{Pisa}{Cecilia Pisa et. al., ''Spherical Photogrammetry for Cultural Heritage San Galgano Abbey and the Roman Theater, Sabratha'',2011.}

        %Reason why Vive is cool
        \bibitem{Soffel}{Fabian Soffel et. al., ''Postural Stability Analysis in Virtual Reality Using the HTC Vive'',2016.}

        %How VR can solve practical problems
        \bibitem{Hilfert}{Thomas Hilfert and Markus K�onig, ''Low-cost virtual reality environment for
engineering and construction'',2016.}

        %VR story telling, interactivity vs influence
        \bibitem{Craven}{Mike Craven et. al., ''Exploiting Interactivity, Influence, Space and Time
to Explore Non-Linear Drama in Virtual Worlds'',2001.}

        %How VR can help understand space
        \bibitem{Rauhoeft}{Greg Rauhoeft et. al., ''Evoking and Assessing Vastness in Virtual Environments'',2015.}

        %Assessing 3D scan quality
        \bibitem{Thorn}{Jacob Thorn et. al., ''Assessing 3D Scan Quality Through Paired-comparisons Psychophysics'',2016.}
        
        %https://www.engadget.com/2016/03/18/htc-vive-an-oral-history/

	     \bibitem{Pi3DScan}{Pi3DScan, ''http://www.pi3dscan.com/'',2016.}

		%	Characterization of a Large, Low-Cost 3D Scanner
		\bibitem{Straub1}{Jeremy Straub. Benjamin Kading. Atif Mohammad and Scott Kerlin, ''Characterization of a Large, Low-Cost 3D Scanner'',2015.}
		
		%Development of a Large, Low-Cost, Instant 3D Scanner
		\bibitem{Straub2}{Jeremy Straub and Scott Kerlin, ''Development of a Large, Low-Cost, Instant 3D Scanner'',2014.}
		
		\bibitem{Volpe}
		Joseph Volpe,
		\emph{A visual history of Valve's work in VR}.
		\newline
		\url{https://www.engadget.com/2015/03/04/a-history-of-valve-vr/},
		2015.
		
		\bibitem{Souppouris}
		Aaron Souppouris,
		\emph{How HTC and Valve built the Vive}.
		\newline
		\url{https://www.engadget.com/2016/03/18/htc-vive-an-oral-history/},
		2016.
		
		\bibitem{Samann}{Mariam Samaan, Marc Pierrot Deseillingy, and Raphae Le Heno, ''Close-Range Photogrammetric Tools for Epigraphic Surveys'',2016.}
    	\end{thebibliography}
\end{document}
